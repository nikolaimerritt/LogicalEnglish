\documentclass[../main.tex]{subfiles}

\begin{document}
\chapter{Introduction}
\section{Abstract}
Language Extensions for code editors are a crucial tool in writing code quickly and without errors. In this project, I create a language extension for the logical, declarative programming language Logical English. The language extension highlights the syntactic and semantic features of Logical English, identifies errors, and generates ``boilerplate'' code to fix them. The language extension uses the Language Server Protocol and is therefore cross-editor. It is evaluated when connected to the Visual Studio Code and Codemirror IDEs.
\section{Motivation}
\subsection{Why a new editor?}
Logical English is a relatively new programming language, first introduced in late 2020 \cite{logical_english} \todo{is this where LE was first introduced?}. Although Logical English has an online editor hosted on the SWISH platform \cite{swish_editor}, the editor is not user-friendly. SWISH is primarily a Prolog editor, and so any Logical English code has to be written in a long string that is, in the same file, passed to a Prolog function to be interpreted. 
\\ \\ 
This has significant drawbacks: since Logical English code is written in a Prolog string, the Logical English content cannot be treated by the editor as a standalone program. This means that it can receive no syntax highlighting, error detection, or code completion features beyond that of a Prolog string. Since these features are essential for productivity \todo{find a source here about IDEs vs Notepad}, a new editor was needed that was custom-built for writing Logical English.

\subsection{Why a Language Server?}
When deciding what type of language extension to create, we surveyed a variety of options. The SWISH platform in which Logical English is currently edited is built on Codemirror, a JavaScript framework for creating web-based code editors. Thus writing the language server entirely in Codemirror was an obvious choice. However, at the time, the SWISH platform was written in Codemirror 5, but the maintainers were considering migrating to Codemirror 6. Prematurely writing the language extension in Codemirror 6 would be too much of a risk, as migration was uncertain, and users would not be able to test the extension until it was migrated. However, using Codemirror 5 would also have been a bad idea, as migration to Codemirror 6 would involve a large number of breaking changes \cite{codemirror_migration}, such as getting the position of text in a document, and making changes to the document, being entirely restructured.
\\ 
\\ 
This prompted us to then consider Monaco. \todo[inline]{Why were we considering Monaco? How would Monaco have worked with SWISH?}.
\\ \\ 
In the end, we found that what we really needed was a language server. This language server would be editor-agnostic, meaning that one language server would be able to connect to any front-end that supports the Language Server Protocol. This includes online editors such as Codemirror 6 \cite{codemirror_6_language_server} and Monaco \cite{monaco_language_server}, along with desktop editors such as Visual Studio Code \cite{vsc_langserver_docs}, Visual Studio \cite{visual_studio_language_server} and IntelliJ \cite{intellij_language_server} some of the most popular code editors \cite{ide_rankings}. I also found that the language server I produced was able to communicate error messages with Codemirror 5. Logical English is still in an early stage of development, and producing a language server would give us the flexibility needed to branch out to all kinds of coding environments.

\subsection{Why a separate Syntax Highlighter?}
Although language servers can mark code for highlighting (and, indeed, mine does), it is common to delegate all the syntactic highlighting to the client. This is done for efficiency reasons. Syntactic highlighting does not need a complex algorithm to parse the document, and it would be a waste of resources to do so: instead, it can be done through identifying parts of the document using regular expressions. 
\\ 
\\
This is commonly done using a TextMate grammars \cite{textmate_grammars_spec} document. This is a JSON document that assigns certain standard labels to sections of the document that match regular expressions. Syntax highlighting using TextMate grammars is supported by all the IDEs mentioned above.
\\ 
\\
\todo[inline]{Talk about how the TextMate grammar marks words, and it is up to the IDEs colour scheme to colour them appropriately.}
Throughout this report, the term ``language extension" will refer to the language server and syntax highlighter together.
\end{document}