\documentclass[../main.tex]{subfiles}

\begin{document}
\section*{Project Specification}
The project will consist of developing two tools for Logical English: a Syntax Highlighter and a Language Server. These two tools are cross-editor, meaning that they can be used with many of the most popular programming editors with minimal configuration.

\subsection*{Syntax Highlighter}
The Syntax Highlighter will identify both micro-features of Logical English such as keywords and variable names, and macro-features such as section headers. It will identify these features using TextMate grammar. This way, the default themes of many popular code editors will recognise and style them. 

\subsection*{Language Server}
The Language Server will allow the user to generate new templates from rules. If a set of rules do not match any existing templates, the Language Server will communicate this to the user. It will allow the user to, at the click of a button, generate a template that matches the rules. 
\\ \\ 
If there is time, we will give the Language Server the feature to alert the user of certain type mismatch errors. The user will be notified of errors where a rule is supplied in the knowledge base with a type that conficts with the corresponding type in the rule's template. To determine whether the one type conficts with the other, the Language Server would consider type inheritance as supported by Logical English.
\\ \\
The language server will communicate with potential language clients using the Langauge Server Protocol. This way, many popular code editors will be able to communicate with the language server.

\section*{Project Implementation Plan}
The Syntax Highlighter will be implemented using TextMate grammar, since this has the widest range of editor support.  The Language Server will be implemented in TypeScript using the \texttt{vscode-languageserver} NPM package, since this package has clear, thorough documentation and describes multiple example language servers. In testing, both the language server and syntax highlighter will be run through a Visual Studio Code language client. This choice is made due to Visual Studio Code's powerful debugging features for language plugins.

\subsection*{Project Timeline}
The timeline for developing and testing these two tools is below. This plan has us completing both the template generation and type error detection features of the Language Server. However, if any large problems arise, we will prioritise solving these over working on type error detection.
\\
\begin{tabularx}{\textwidth}{|l|X|}
    \hline
    6th June - 10th June & 
    Learn Regular Expressions. \newline
    Write a TextMate grammar for Logical English.
    \\
    \hline
    13th June - 17th June & 
    Using the Visual Studio Code documentation \cite[]{vsc_langserver_features}, create a proof-of-concept language server with dummy error highlighting, warning highlighting, and code generation.
    \\ \hline
    20th June - 25th June & 
    Convert Logical English templates to a suitable TypeScript representation. \newline
    Using this representation, determine whether a Logical English rule conforms to a template.
    \\ \hline
    27th June - 8th July & 
    Create a template from first two, then arbitrarily many, rules.
    \\ \hline
    11th July - 23rd July & 
    Create a Logical English Type representation in TypeScript. \newline
    Use this type representation in giving types to the arguments of templates. Use this type representation to consider types in determining whether a rule conforms to a template.
    \\ \hline
\end{tabularx}
\end{document}